%\documentclass[10pt,landscape]{article}
\documentclass[paper=letter,fontsize=3mm]{scrartcl}

\usepackage{multicol}
\usepackage{calc}
\usepackage{ifthen}
\usepackage[landscape]{geometry}
\usepackage{hyperref}
\usepackage{centernot}
\usepackage{amsmath,amssymb,amsthm}

\linespread{1}

% Math Operators
\DeclareMathOperator{\Var}{Var}
\DeclareMathOperator{\Cov}{Cov}
\DeclareMathOperator{\Corr}{Corr}
\DeclareMathOperator{\E}{E}
\DeclareMathOperator{\Prob}{P}
\DeclareMathOperator{\diag}{diag}
\DeclareMathOperator{\rank}{rank}
\DeclareMathOperator{\tr}{tr}
\DeclareMathOperator*{\argmin}{arg\,min}
\DeclareMathOperator*{\argmax}{arg\,max}

% Math Commands 
\newcommand\given[1][]{\:#1\vert\:}
\newcommand{\ind}{\stackrel{\text{ind}}{\sim}}
\newcommand{\iid}{\stackrel{\text{iid}}{\sim}}
\newcommand{\eqdist}{\stackrel{\text{d}}{=}}
\newcommand{\convdist}{\stackrel{\text{d}}{\longrightarrow}}
\newcommand{\convprob}{\stackrel{\text{p}}{\longrightarrow}}
\newcommand{\convas}{\stackrel{\text{a.s.}}{\longrightarrow}}
\newcommand{\convL}[1]{\stackrel{\mathcal{L}_{#1}}{\longrightarrow}}
\newcommand{\Norm}{\mathcal{N}} 
\newcommand{\Borel}{\mathcal{B}}
\newcommand{\eps}{\varepsilon}
\newcommand{\R}{\mathbb{R}}
\newcommand{\Q}{\mathbb{Q}}
\newcommand{\C}{\mathbb{C}}
\newcommand{\N}{\mathbb{N}}
\newcommand{\Z}{\mathbb{Z}}
\newcommand\indicate[1]{\mathbb{I}_{ #1 }}
\newcommand\abs[1]{\left| #1 \right|}
\newcommand\norm[1]{\left\lVert #1 \right\rVert}
\newcommand\inner[1]{\left\langle #1 \right\rangle}
\newcommand\set[1]{\left\{ #1 \right\}}

% To make this come out properly in landscape mode, do one of the following
% 1.
%  pdflatex latexsheet.tex
%
% 2.
%  latex latexsheet.tex
%  dvips -P pdf  -t landscape latexsheet.dvi
%  ps2pdf latexsheet.ps

% This sets page margins to .5 inch if using letter paper, and to 1cm
% if using A4 paper. (This probably isn't strictly necessary.)
% If using another size paper, use default 1cm margins.
\ifthenelse{\lengthtest { \paperwidth = 11in}}
	{ \geometry{top=.2in,left=.25in,right=.25in,bottom=.2in} }
	{\ifthenelse{ \lengthtest{ \paperwidth = 297mm}}
		{\geometry{top=1cm,left=1cm,right=1cm,bottom=1cm} }
		{\geometry{top=1cm,left=1cm,right=1cm,bottom=1cm} }
	}

% Turn off header and footer
\pagestyle{empty}
 

% Redefine section commands to use less space
\makeatletter
\renewcommand{\section}{\@startsection{section}{1}{0mm}%
                                {-1ex plus -.5ex minus -.2ex}%
                                {0.5ex plus .2ex}%x
                                {\normalfont\large\bfseries}}
\renewcommand{\subsection}{\@startsection{subsection}{2}{0mm}%
                                {-1explus -.5ex minus -.2ex}%
                                {0.5ex plus .2ex}%
                                {\normalfont\normalsize\bfseries}}
\renewcommand{\subsubsection}{\@startsection{subsubsection}{3}{0mm}%
                                {-1ex plus -.5ex minus -.2ex}%
                                {1ex plus .2ex}%
                                {\normalfont\small\bfseries}}
\makeatother


% Don't print section numbers
\setcounter{secnumdepth}{0}


\setlength{\parindent}{0pt}
\setlength{\parskip}{0pt plus 0.01ex}


% -----------------------------------------------------------------------

\begin{document}

\raggedright
\scriptsize
\begin{multicols*}{3}


% multicol parameters
% These lengths are set only within the two main columns
%\setlength{\columnseprule}{0.25pt}
\setlength{\premulticols}{.05pt}
\setlength{\postmulticols}{.05pt}
\setlength{\multicolsep}{.05pt}
\setlength{\columnsep}{.05pt}

\subsection*{Classes of Sets}

A collection $\mathcal{F}$ of subsets of $\Omega \ne \emptyset$ is an \textbf{algebra} if
\begin{enumerate}
\item  $\Omega \in \mathcal{F}$,
\item $A \in \mathcal{F} \implies A^c \in \mathcal{F}$,
\item $A,B \in \mathcal{F} \implies A \cup B \in \mathcal{F}$ (equivalently, $A \cap B \in \mathcal{F}$).
\end{enumerate}

A collection $\mathcal{F}$ of subsets of $\Omega$ is a \textbf{$\sigma$-algebra} if
\begin{enumerate}
\item $\mathcal{F}$ is an algebra,
\item $A_1, A_2, \dots \in \mathcal{F} \implies \bigcup_{n=1}^\infty A_n \in \mathcal{F}$.
\end{enumerate}

If $\mathcal{F}_\theta, ~\theta \in \Theta$ is a collection of $\sigma$-algebras on $\Omega$, then $\mathcal{G} = \bigcap_{\theta\in\Theta}\mathcal{F}_\theta$ is a $\sigma$-algebra but $\bigcup_{\theta\in\Theta}\mathcal{F}_\theta$ may not be. \\\medskip

If $\mathcal{A}$ is a collection of subsets of $\Omega$, then the $\sigma$-algebra generated by $\mathcal{A}$ is 
$$\sigma\langle\mathcal{A}\rangle = \bigcap_{{\mathcal{F} \text{ is } \sigma\text{-algebra} \atop \mathcal{A} \subset \mathcal{F}}} \mathcal{F}.$$

A class $\mathcal{C}$ of subsets of $\Omega$ is a \textbf{$\pi$-class} if
\begin{enumerate}
\item $A, B \in \mathcal{C} \implies A \cap B \in \mathcal{C}$.
\end{enumerate}

A class $\mathcal{L}$ of subsets of $\Omega$ is a \textbf{$\lambda$-class} if
\begin{enumerate}
\item  $\Omega \in \mathcal{L}$,
\item $A \in \mathcal{L} \implies A^c \in \mathcal{L}$,
\item $A_1, A_2, \dots \in \mathcal{L} \text{ \bf{disjoint} } \implies \bigcup_{n=1}^\infty A_n \in \mathcal{L}$.
\end{enumerate}

$\lambda$-class $\centernot\implies \sigma$-algebra. \\
$\pi$-class and $\lambda$-class $\implies \sigma$-algebra (but not converse). \\ \medskip

\textbf{Dynkin's $\pi-\lambda$ Theorem}:
On a set $\Omega$, if $\mathcal{C}$ is a $\pi$-class and $\mathcal{L}$ is a $\lambda$-class such that $\mathcal{C} \subset \mathcal{L}$, then $\sigma\langle \mathcal{C} \rangle \subset \mathcal{L}$. \\\medskip

\subsection*{Product Spaces}
If $\mathcal{F}$ is a $\sigma$-algebra on a nonempty set $\Omega$, then $(\Omega, \mathcal{F})$ is a \textbf{measurable space}. 
If $(\Omega_i, \mathcal{F}_i), ~ i = 1, 2, \dots, n$ are measurable spaces, the $n$-dimensional \textbf{product space} is 
$$\prod_{i=1}^n \Omega_i = \set{(\omega_1, \omega_2, \dots, \omega_n): \omega_i \in \Omega_i, ~1 \le i \le n}.$$

If $A_i \subset \Omega_i$, the $n$-dimensional \textbf{rectangle} is 
$$\prod_{i=1}^n A_i = \set{(\omega_1, \omega_2, \dots, \omega_n): \omega_i \in A_i, ~1 \le i \le n}.$$

The $n$-dimensional \textbf{product $\sigma$-algebra} is
$$\prod_{i=1}^n \mathcal{F}_i = \sigma\left\langle \set{\prod_{i=1}^n A_i: A_i \in \mathcal{F}_i, ~ 1 \le i \le n }\right\rangle.$$

\subsection*{Measures}
An extended real-valued function on a class of subsets of a nonempty set $\Omega$ is a \textbf{set function}. 
A set function $\mu$ on an algebra $\mathcal{F}$ on $\Omega$ is a \textbf{measure} if
\begin{enumerate}
\item $\mu(A) \in [0,\infty], \quad \forall A \in \mathcal{F}$,
\item $\mu(\emptyset) = 0$, 
\item $A_1, A_2, \dots \in \mathcal{L}$ \textbf{disjoint} with  $\bigcup_{n=1}^\infty A_n \in \mathcal{F}$, then
$$\mu\left(\bigcup_{n=1}^\infty A_n\right) = \sum_{n=1}^\infty \mu(A_n).$$
\end{enumerate}

If $\mu(\Omega) = 1$, then $\mu$ is a \textbf{probability measure}. If $\exists B_1, B_2 \in \mathcal{F}$ such that $\Omega = \bigcup_{n=1}^\infty B_n$ and $\mu(B_i) < \infty, ~\forall i \ge 1$, then $\mu$ is \textbf{$\sigma$-finite}. \\\medskip

\textbf{Extensions of Measures}: If $\mu$ is a $\sigma$-finite measure on an algebra $\mathcal{F}$, then there is a \emph{unique} extension of $\mu$ to $\sigma\langle \mathcal{F}\rangle$. \\\medskip

\textbf{Monotonicity}: If $A, B \in \mathcal{F}$ such that $A \subset B$, then $P(A) \le P(B)$. \\\medskip
\textbf{Inclusion-exclusion formula}: If $A_1,A_2,\dots,A_n \in \mathcal{F}$, then
$$P\left( \bigcup_{i=1}^n A_i\right) = \sum_{i=1}^n P(A_i) - \sum_{1\le i < j \le n} P(A_i \cap A_j) + \cdots + (-1)^{n-1}P(A_1 \cap \dots \cap A_n).$$

\textbf{Countable subadditivity}: If $A_1,A_2,\dots \in \mathcal{F}$ such that $\bigcup_{n=1}^\infty A_n \in \mathcal{F}$, then
$$P\left( \bigcup_{n=1}^\infty A_n\right) \le \sum_{n=1}^\infty P(A_n).$$

\textbf{Monotone continuity from below} (mcfb): If $A, A_1, A_2, \dots \in \mathcal{F}$ such that $A_n \uparrow A$ (i.e., $A_n \subset A_{n+1}$ and $A = \bigcup A_n$), then $P(A_n) \uparrow P(A)$. \\\medskip

\textbf{Monotone continuity from above} (mcfa): If $A, A_1, A_2, \dots \in \mathcal{F}$ such that $A_n \downarrow A$ (i.e., $A_n \supset A_{n+1}$ and $A = \bigcap A_n$), then $P(A_n) \downarrow P(A)$. \\\medskip

\subsection*{Measurable Transformations}

If $(\Omega_1, \mathcal{F}_1)$, $(\Omega_2, \mathcal{F}_2)$ are measurable spaces, then $T: \Omega_1 \to \Omega_2$ is \textbf{$\langle \mathcal{F}_1, \mathcal{F}_2\rangle$-measurable} if
$$T^{-1}(A) \equiv \set{\omega \in \Omega_1: T(\omega) \in A} \in \mathcal{F}_1 (\text{equivalently}, T^{-1}(\mathcal{F}_2) \subset \mathcal{F}_1).$$

Let $(\Omega, \mathcal{F}, P)$ be a psp. Then, $X: \Omega \to \R$ is a \textbf{r.v.\@} if it is $\langle \mathcal{F}, \Borel(\R)\rangle$-measurable. \\\medskip

Checking for measurability: $T: \R \to \R \text{ is } \langle\Borel(\R), \Borel(\R)\rangle\text{-measurable }$ iff $T^{-1}(-\infty, r) = \set{\omega \in \R: T(\omega) < r} \in \Borel(\R) ~\forall r \in \R$. \\\medskip

If $T_i$ is $\langle \mathcal{F}_i, \mathcal{F}_{i+1}\rangle$-measurable for $i = 1,2$, then the composition $T = T_1 \circ T_2 = T_1(T_2(\cdot))$ is $\langle \mathcal{F}_1, \mathcal{F}_3\rangle$-measurable. \\\medskip

If $f: \R^k \to \R^p$ is continuous, then $f$ is $\langle\Borel(\R), \Borel(\R)\rangle$-measurable. \\\medskip

If $f_1, f_2, \dots, f_n$ are each $\langle \mathcal{F},\Borel(\R) \rangle$-measurable transformations from $\Omega$ to $\R$, then $f = (f_1, f_2, \dots, f_n)'$ is $\langle\mathcal{F},\Borel(\R)\rangle$-measurable. Also, $\sum_{i=1}^n f_i$, $\prod_{i=1}^nf_i$, $\sup_n f_n$, $\inf_n f_n$, $\limsup f_n$, $\liminf f_n$, and $\indicate{\set{\omega\in\Omega:\lim_{n\to\infty}f_n(\omega) \text{ finitely exists}}} \lim_{n\to\infty}f_n$ are all $\langle\mathcal{F}, \Borel(\R)\rangle$-measurable. \\\medskip

\subsection*{Induced Measures \& Distribution Functions}
If $(\Omega_1, \mathcal{F}_1)$, $(\Omega_2, \mathcal{F}_2)$ are measurable spaces and $T: \Omega_1 \to \Omega_2$ is $\langle \mathcal{F}_1, \mathcal{F}_2\rangle$-measurable, then for any measure $\mu$ on $(\Omega_1, \mathcal{F}_1)$, the set function $\mu T^{-1}$ defined by
$$\mu T^{-1}(A) = \mu \circ T^{-1}(A) = \mu(T^{-1}(A)), \quad \forall A \in \mathcal{F}_2$$
is a measure on $\mathcal{F}_2$ and is called the \textbf{measure induced} by $T$ under $\mu$ on $\mathcal{F}_2$. \\\medskip

For a r.v.\@ $X$ on a psp $(\Omega, \mathcal{F}, P)$, the probability \textbf{distribution} of $X$ (or the law of $X$), denoted $P_X$,  is the induced measure of $X$ under $P$ on $\Borel(\R)$. That is,
$$P_X(A) = P(X^{-1}(A)) = P(\set{\omega \in \Omega: X(\omega) \in A}) = P(X \in A), \quad A \in \Borel(\R).$$

The \textbf{cumulative distribution function} (cdf) of a r.v.\@ $X$ is
$$F(x) = P_X((-\infty,x]) = P(X \le x), \quad x \in \R.$$

If $F$ is the cdf of a r.v.\@ $X$, then 
\begin{enumerate}
\item $F$ is right continuous: if $x_n \downarrow x_0$ and $x_n \ge x$ then $F(x_n) = P_X((-\infty,x_n]) \downarrow P_X((-\infty,x_0]) = F(x_0)$ by mcfa,
\item $F$ is monotone nondecreasing: if $x \le y \implies (-\infty, x] \subset (-\infty, y]$ then $F(x) = P_X((-\infty,x]) \le P_X((-\infty,y]) =F(y)$ by monotonicity,
\item $\lim_{x\to\infty} F(x) = 1$ and $\lim_{x\to-\infty} F(x) = 0$: show using argument similar to (1).
\end{enumerate}

If $F: \R \to [0,1]$ satisfies (1)-(3) above, then there exists a r.v.\@ $X$ on a psp $(\Omega, \mathcal{F}, P)$ such that $F$ is the cdf of $X$. \\\medskip

\subsection*{Integrals}
Any extended real-valued function $f$ defined on $\Omega$ can be decomposed into positive $f^+ = \indicate{f\ge0}f$ and and negative $f^- = -\indicate{f<0}f$ parts. Then, $f = f^+ - f^-$ and $\abs{f} = f^+ + f^-$. \\\medskip

A measurable function $f: \Omega \to \overline{\R}$ on $(\Omega, \mathcal{F}, \mu)$ is \textbf{$\mu$-integrable} if $\int_\Omega \abs{f} d\mu < \infty$. If $\mu(\abs{f} = \infty) > 0$, then $\int_\Omega \abs{f} d\mu \equiv \infty$. Note $\abs{f}$ is $\mu$-integrable iff both $f^+$ and $f^-$ are $\mu$-integrable. \\\medskip

If at least one of $\int_\Omega f^+ d\mu$, $\int_\Omega f^- d\mu$ is $< \infty$, then $\int_\Omega f d\mu \equiv \int_\Omega f^+ d\mu - \int_\Omega f^- d\mu$ and we say $\int_\Omega f d\mu$ exists. \\\medskip

If $f: \Omega \to \overline{\R}$ is nonnegative and $\mu$-integrable, then $f < \infty$ a.e.($\mu$). If $f: \Omega \to \overline{\R}$ is nonnegative, then $\int_\Omega fd\mu = 0$ iff $f = 0$ a.e.($\mu$). \\\medskip

If $(\Omega, \mathcal{F}, \mu)$ is a msp and $f:\Omega \to \R$ is measurable and either $\mu$-integrable or nonnegative. Then, for any $A \in \mathcal{F}$, 
$$\int_A fd\mu = \int_\Omega \indicate{A}fd\mu.$$

Using the DCT (see next section), it can be shown that for \textbf{disjoint} $A_1, A_2, \dots \in \mathcal{F}$ and measurable $f: \Omega \to \R$ either $\mu$-integrable or nonnegative, then 
$$\int_\Omega f \indicate{\bigcup_{n=1}^\infty A_n}d\mu = \sum_{n=1}^\infty \int_{A_i} f d\mu.$$

\subsection*{Convergence Theorems}
\textbf{Monotone Convergence Theorem} (MCT): If $f_n: \Omega \to \overline{\R}$ is an increasing sequence of nonnegative measurable functions, i.e., $f_n(\omega) \le f_{n+1}(\omega) ~\forall \omega \in \Omega$ and $f_n(\omega) \uparrow f(\omega) \text{ a.e.}(\mu)$, then $\int_\Omega f_n d\mu \uparrow \int_\Omega f d\mu$. That is,
$$\int_\Omega f d\mu = \int_\Omega \lim_{n\to\infty} f_nd\mu = \lim_{n\to\infty}\int_\Omega f_nd\mu.$$

\textbf{Fatou's Lemma}:
If $f_n: \Omega \to \overline{\R}$ is a sequence of nonnegative functions, then $$\int_\Omega \liminf_{n\to\infty} f_n d\mu \le \liminf_{n\to\infty}  \int_\Omega f_n d\mu.$$

\textbf{Dominated Convergence Theorem} (DCT): 
Suppose (1) $g: \Omega \to \overline{\R}$ is a nonnegative, $\mu$-integrable function; (2) $\abs{f_n} \le g \text{ a.e.}(\mu) ~\forall n \ge 1$; and (3) $f_n \to f \text{ a.e.}(\mu)$. Then, $f$ is $\mu$-integrable and $$\lim_{n\to\infty}\int_\Omega \abs{f_n - f} d\mu = 0 \text{ and } \lim_{n\to\infty} \int_\Omega f_nd\mu = \int_\Omega fd\mu.$$

\emph{Result under weaker conditions, UI and $\mu(\Omega) < \infty$}: If $(\Omega, \mathcal{F}, \mu)$ is a msp with $\mu(\Omega) < \infty$ and $f,f_n: \Omega \to \overline{R}$ are measurable such that $f_n \to f$ a.e.($\mu$) and $\set{f_n: n \ge1}$ is UI (see next section), then $f$ is $\mu$-integrable and $$\lim_{n\to\infty} \int_\Omega f_nd\mu = \int_\Omega f d\mu.$$

Note: for a nonnegative measurable real-valued function $f$ on a msp $(\Omega, \mathcal{F}, \mu)$, $\nu(A) = \int_A fd\mu ~\forall A \in \mathcal{F}$ is a measure on $(\Omega, \mathcal{F})$ and $f$ is a density of the measure $\nu$. \\\medskip

\textbf{Scheffe's Theorem}: Let $(\Omega, \mathcal{F}, \mu)$ be a msp and for $n \ge 0$, let $\nu_n(A) = \int_A f_n d\mu ~\forall A \in \mathcal{F}$ be finite measures on $\mathcal{F}$ with densities $f_n \ge 0$. If $\nu_n(\Omega) = \nu_0(\Omega) < \infty$ for all $n \ge 1$ and $f_n \to f$ a.e.($\mu$), then $$\lim_{n\to\infty} \int_\Omega \abs{f_n-f_0}d\mu = 0.$$
Also, the ``total variation norm'' tends to zero: 
$$\sup_{A\in\mathcal{F}} \abs{\nu_n(A) - \nu_0(A)} =\frac{1}{2}\int_\Omega \abs{f_n - f_0}d\mu \to 0 \text{ as } n \to \infty.$$

\subsection*{Uniform Integrability}
Recall if $f: \Omega \to \R$ is $\mu$-integrable, then by the DCT $$\lim_{n\to\infty}\int_{\abs{f}>n} \abs{f}d\mu = \lim_{n\to\infty}\int_\Omega \indicate{\abs{f}>n}\abs{f}d\mu=0.$$

A family of $\mu$-integrable functions $\set{f_\lambda: \lambda \in \Lambda}$ on a msp $(\Omega, \mathcal{F}, \mu)$ is \textbf{uniformly integrable} (UI) w.r.t.\@ $\mu$ if 
$$\sup_{\lambda \in \Lambda} \int_{\abs{f_\lambda}>t}\abs{f_\lambda}d\mu \to 0 \text{ as } t \to \infty.$$

Suppose $\mathcal{A} \equiv \set{f_\lambda: \lambda \in \Lambda}$ is a collection of $\mu$-integrable functions on a msp $(\Omega, \mathcal{F}, \mu)$. Then,
\begin{itemize}
\item if $\Lambda$ is a finite set, then $\mathcal{A}$ is UI,
\item if $\exists \eps > 0$ such that $\sup\set{\int\abs{f_\lambda}^{1+\eps}d\mu:\lambda \in \Lambda} < \infty$, then $\mathcal{A}$ is UI,
\item if $\abs{f_\lambda} \le f$ a.e.($\mu$) and $\int f d\mu < \infty$, then $\mathcal{A}$ is UI,
\item if $\mathcal{A}$ is UI and $\mu(\Omega) < \infty$, then $\exists M > 0$ such that $\sup\set{\int\abs{f_\lambda}d\mu:\lambda \in \Lambda} \le M$,
\item if $\set{f_\lambda: \lambda \in \Lambda}$ and $\set{g_\lambda: \lambda \in \Lambda}$ are both UI, then $\set{f_\lambda + g_\lambda: \lambda \in \Lambda}$ is also UI.
\end{itemize}

\subsection*{Independence}
Let $(\Omega, \mathcal{F}, P)$ be a psp and $I$ be a set of indices. 
\begin{itemize}
\item
A collection $A_i, ~i \in I$ of sets in $\mathcal{F}$ are \textbf{ind} if $\forall i_1, i_2, \dots, i_n \in I$ distinct indices and fixed $1 \le n < \infty$,
$$P\left(\bigcap_{j=1}^n A_{i_j}\right) = \prod_{j=1}^n P\left(A_{i_j}\right).$$
Note the above has $2^n -n -1$ independence conditions!

\item
Suppose $\mathcal{G}_i \subset \mathcal{F}$ is a collection of measurable sets for each $i \in I$. Then the family of sets $\set{\mathcal{G}_i : i \in I}$ is called \textbf{ind} if any possible collection $\set{A_i: i \in I}$ of sets are ind, where $\set{A_i:i \in I}$ is formed by choosing an arbitrary set $A_i$ from $\mathcal{G}_i$ for each $i \in I$. That is, $\forall i_1, i_2, \dots, i_n \in I$ distinct indices, fixed $1 \le n < \infty$ and $\forall A_{i_1}, A_{i_2}, \dots, A_{i_n} \in \mathcal{G}_{i_n}$,
$$P\left(\bigcap_{j=1}^n A_{i_j}\right) = \prod_{j=1}^n P\left(A_{i_j}\right).$$
Note the family of sets $\set{\mathcal{G}_i: i \in I}$ are ind iff for each finite $T \subset I$, $\set{\mathcal{G}_i:i \in T}$ are ind.

\item
A collection of r.v.\@'s $X_i, ~i \in I$ on $(\Omega, \mathcal{F}, P)$ are \textbf{ind} if the family $\set{\sigma\langle X_i \rangle: i \in I}$ is ind, where 
$$sigma\langle X_i \rangle = \set{X_i^{-1}(B): B \in \Borel(\R)} = X_i^{-1} (\Borel(\R))$$
is the $\sigma$-algebra generated by $X_i$. That is, $\forall i_1, i_2, \dots, i_n \in I$ distinct indices, fixed $1 \le n < \infty$ and $\forall B_{i_1}, B_{i_2}, \dots, B_{i_n} \in \Borel(\R)$,
$$P\left(X_{i_1} \in B_{i_1}, X_{i_2} \in B_{i_2}, \dots, X_{i_n} \in B_{i_n}\right) = \prod_{j=1}^n P\left(X_{i_j}\in B_{i_j}\right).$$
In terms of distribution functions, $X_i, i \in I$ are ind iff  $\forall x_1, x_2, \dots, x_n \in \R$ and $\forall i_1, i_2, \dots, i_n \in I$ distinct indices,
$$P\left(X_{i_1} \le x_1, X_{i_2} \le x_2, \dots, X_{i_n} \le x_n \right) = \prod_{j=1}^n P\left(X_{i_j}\le x_j\right).$$
\end{itemize}

\textbf{Independence of generated $\sigma$-algebras}: If $(\Omega, \mathcal{F}, P)$ is a psp, $\mathcal{G}_i \subset \mathcal{F}$ is a $\pi$-class for each index $i \in I$, and the family $\set{\mathcal{G}_i: i \in I}$ is ind. Then, the family 
$\set{\sigma\langle\mathcal{G}_i\rangle: i \in I}$ is ind. \\\medskip

\subsection*{Borel-Cantelli Lemmas}
If $(\Omega, \mathcal{F})$ be a measurable space and $A_1, A_2, \dots \in \mathcal{F}$, then
$$\limsup_{n\to\infty} A_n = \overline{\lim} A_n = \bigcap_{k=1}^\infty\bigcup_{n=k}^\infty A_n \in \mathcal{F},$$
$$\liminf_{n\to\infty} A_n = \underline{\lim} A_n = \bigcup_{k=1}^\infty\bigcap_{n=k}^\infty A_n \in \mathcal{F}.$$
Also,
$$\liminf_{n\to\infty} A_n \subset \limsup_{n\to\infty} A_n,$$
$$(\text{``$A_n$ i.o''})^c = \left(\overline{\lim}A_n\right)^c = \underline{\lim}A_n^c = \text{``$A_n^c$  eventually,''}$$
$$(\text{``$A_n$ eventually''})^c = \left(\underline{\lim}A_n\right)^c  = \overline{\lim}A_n^c = \text{``$A_n^c$  i.o.''}$$

\textbf{Borel-Cantelli Lemma}: Let $(\Omega, \mathcal{F}, P)$ be a psp and $A_1, A_2, \dots \in \mathcal{F}$. 
\begin{enumerate}
\item If $\sum_{n} P(A_n) < \infty$, then $P\left( \overline{\lim} A_n\right) = 0$.
\item If $\set{A_n}$ are ind and $\sum_{n} P(A_n) = \infty$, then $P\left( \overline{\lim} A_n\right) = 1$.
\end{enumerate}

\textbf{Borel 0-1 Law}: If $A_1, A_2, \dots$ are ind events, then
$$P(\overline{\lim} A_n) = P(A_n \text{ i.o.}) = 
\left\{ \begin{array}{ll}
 0 &\mbox{ iff $\sum_{n} P(A_n) < \infty$,} \\
 1 &\mbox{ iff $\sum_{n} P(A_n) = \infty$.}
       \end{array} \right.
$$

\subsection*{Tail Events \& K's 0-1 Law}
The \textbf{tail $\sigma$-algebra} of a sequence of r.v.\@'s $X_1, X_2, \dots$ on a psp $(\Omega, \mathcal{F}, P)$ is
$$\mathcal{T} = \bigcap_{n=1}^\infty \sigma \left\langle \set{X_j: j \ge n} \right\rangle,$$
where $ \sigma \left\langle \set{X_j: j \ge n} \right\rangle =  \sigma \left\langle \set{X_j^{-1}:B \in \Borel(\R),  j \ge n} \right\rangle$ is the $\sigma$-algebra generated by $X_j, j \ge n$. \\\medskip

Any set (event) $A \in \mathcal{T}$ is a \textbf{tail event}. \\\medskip

An extended real-valued r.v.\@ $T: \Omega \to \overline{\R} = \R \cup \set{-\infty, \infty}$ is a \textbf{tail r.v.\@} if $T$ is $\langle \mathcal{F}, \Borel(\overline{\R})\rangle$-measurable. That is, $\forall r \in \R, ~ T^{-1}([-\infty, r)) = \set{\omega \in \Omega: T(\omega) < r} \in \mathcal{T}$. \\\medskip

For fixed, arbitrary $m \ge 1$, a tail event or r.v.\@ is only determined by $X_n,  n \ge m$ (so changing finitely man r.v.\@'s does not effect $\mathcal{T}$). \\\medskip

Examples: 
\begin{itemize}
\item $\overline{\lim} X_n$, $\underline{\lim}X_n$ are extended real-valued tail r.v.\@'s,
\item $\set{\lim X_n \text{ finitely exists}}$, $\{\overline{\lim} X_n \ge r\}$, $\set{\underline{\lim} X_n < r}$ are tail events. 
\end{itemize}

\textbf{Kolmogorov's 0-1 Law}: Tail events of a sequence $X_1, X_2, \dots$ of ind r.v.\@'s have probabilities 0 or 1. That is, if $\set{X_n}_{n\ge1}$ are ind and $A \in \mathcal{T} = \bigcap_{n} \sigma \langle \set{X_j: j \ge n}\rangle$, then $P(A) \in \set{0,1}$. \\\medskip

\textbf{Corollary}: For a psp $(\Omega, \mathcal{F}, P)$ and tail $\sigma$-algebra $\mathcal{T}$ defined by a sequence $X_1, X_2, \dots$ of ind r.v.\@'s, if $T: \Omega \to \overline{\R}$ is a tail r.v.\@ (that is, $T$ is $\langle \mathcal{T}, \Borel(\overline{\R})\rangle$-measurable), then $T$ is degenerate. That is, $\exists c \in \overline{\R}$ such that $P(T = c) = 1$. \\\medskip

\subsection*{Convergence of r.v.\@'s}
A sequence of r.v.\@'s $X_1, X_2, \dots$ on a psp $(\Omega, \mathcal{F}, P)$ \textbf{converge almost surely} to a r.v.\@ $X_0$ on $(\Omega, \mathcal{F}, P)$ if
$$P\left(\set{\lim_{n\to\infty} X_n(\omega) = X_0(\omega)}\right) = 1.$$
TFAE:
\begin{itemize}
\item $X_n \convas 0$,
\item $P(\abs{X_n} > \eps \text{ i.o}) = 0$ for all $\eps > 0$,
\item $\sup_{j\ge n} \abs{X_j - X_0} \convprob 0$ as $n\to\infty$,
\item $\lim_{n\to\infty} P\left(\bigcap_{j=n}^\infty \left[ \abs{X_j - X_0} \le \eps\right]\right) = 1$ for all $\eps > 0$. 
\end{itemize}

A sequence of r.v.\@'s $X_1, X_2, \dots$ on a psp $(\Omega, \mathcal{F}, P)$ \textbf{converge in probability} to a r.v.\@ $X_0$ on $(\Omega, \mathcal{F}, P)$ if
$$\lim_{n\to\infty} P\left(\abs{X_n - X_0} > \eps\right) = 0, \quad \forall \eps > 0.$$
TFAE:
\begin{itemize}
\item $X_n \convprob 0$,
\item $\sup_{m\ge n}\left( \abs{X_m - X_n} > \eps\right) \to 0$ as $n\to\infty$ for all $\eps > 0$,
\item $\forall \set{n_j}$ of $\set{X_n}$,  $\exists \{n_{j_k}\}$ such that $X_{n_{j_k}} \convas X_0$.
\end{itemize}

For a sequence of r.v.\@'s $X_1, X_2, \dots$ on a psp $(\Omega, \mathcal{F}, P)$,
\begin{enumerate}
\item if $X_n \convas X_0$  and $g: \R \to \R$ is continuous, then $g(X_n) \convas g(X_0)$,
\item if $X_n \convprob X_0$  and $g: \R \to \R$ is continuous, then $g(X_n) \convprob g(X_0)$.
\end{enumerate}

A sequence $X_1, X_2, \dots$ of 
$$\mathcal{L}_r(\Omega, \mathcal{F}, P) \equiv \set{\text{measurable } X \in \R: \int_\Omega \abs{X}^r dP < \infty}$$ 
functions \textbf{converges in $\mathcal{L}_r$} to a measurable function $X$ if 
$$\lim_{n\to\infty} \int_\Omega \abs{X_n - X}^r dP = 0.$$

If $X \in \mathcal{L}_r$, then $t^r P(\abs{X} > t) \to 0$ as $t \to \infty$, that is, $\uparrow r \implies$ faster convergence. If $\exists p \in (0,\infty)$ such that $t^p P(\abs{X} > t) \to 0$, then $X \in \mathcal{L}_r ~\forall r \in (0, p)$. \\\medskip

If the r.v.\@'s $X_1, X_2, \dots \in \mathcal{L}_r$, then $\exists X \in \mathcal{L}_r$ such that $X_n \convL{r} X$ iff 
$$\sup_{m\ge n} \E\abs{X_m-X_n}^r \to 0 \text{ as } n \to \infty.$$

For a r.v.\@ $X$, the fixed, real-valued $m(X)$ is a \textbf{median} if $P(X \ge m(X)) \ge 1/2$ and $P(X \le m(X)) \ge 1/2$. Can be defined as $\inf\set{x \in \R: P(X \le x) \ge 1/2}$. If $P(\abs{X} \ge c) < \eps \le 1/2$ then $\abs{m(X)} \le c$. \\\medskip

\textbf{Levy's Inequality}: If $X_1, X_2, \dots, X_n$ are ind r.v.\@'s on a psp $(\Omega, \mathcal{F}, P)$ and $S_j = \sum_{i=1}^j X_i, 1 \le j \le n$, then $\forall \eps > 0$,
\begin{enumerate}
\item $P\left( \max_{1\le j \le n}\left[ S_j - m(S_j-S_n)\right] \ge \eps\right) \le 2P(S_n \ge \eps)$,
\item $P\left( \max_{1\le j \le n}\left[ S_j - m(S_j-S_n)\right] \ge \eps\right) \le 2P(\abs{S_n} \ge \eps)$.
\end{enumerate}

\textbf{Levy's Theorem}: If $X_1, X_2, \dots$ are ind r.v.\@'s on a psp $(\Omega, \mathcal{F}, P)$ and  $S_n = \sum_{i=j}^n X_j, n\ge1$, then $S_n$ converges a.s.$(P)$ iff $S_n$ converges in probability. \\\medskip

\textbf{Khintchine-Kolmogorov Convergence Theorem}: If $X_1, X_2, \dots$ are ind r.v.\@'s on a psp $(\Omega, \mathcal{F}, P)$ with $\E(X_n) = 0$ and $\E(X_n^2) < \infty$ for all $n \ge 1$ and $\sum_n \E(X_n^2) < \infty$, then $S_n=\sum_{i=j}^n X_j, n\ge1$ converges a.s.$(P)$ and in $\mathcal{L}_2$ to some random variable $S = \sum_n X_n$. Also, $\E(S) = 0$, $\E(S^2) = \sum_n E(X_n^2)$. \\\medskip

\textbf{Corollary}: If $X_1, X_2, \dots$ are ind r.v.\@'s on a psp $(\Omega, \mathcal{F}, P)$ with $\sum_n \E(X_n) < \infty$ and $\sum_n \sigma^2_{X_n} < \infty$, then $S_n = \sum_{j=1}^n X_j, n\ge1$ converges a.s.$(P)$ to $S = \sum_n X_n$. \\\medskip

Two sequences of r.v.\@'s $\set{X_n}$ and $\set{Y_n}$ are \textbf{tail equivalent} if $\sum_n P(X_n \ne Y_n) < \infty$. If $\set{X_n}$ and $\set{Y_n}$ are tail equivalent, then
\begin{itemize}
\item By Borel-Cantelli, $P\big(\overline{\lim}(X_n \ne Y_n)\big) = 0 \implies P\big(X_n = Y_n \text{ for large } n\big) = 1$, 
\item $S_n = \sum_{j=1}^n X_j \convas S \iff S'_n = \sum_{j=1}^n Y_j \convas S'$,
\item If $b_n \to \infty$, then $$\frac{\sum_{j=1}^n X_j}{b_n} \convas 0 \iff \frac{\sum_{j=1}^n Y_j}{b_n} \convas 0.$$
\end{itemize}

\textbf{Berry-Esseen Lemma}: If $X_1, X_2, \dots, X_n$ are ind r.v.\@'s with $\E(X_i) = 0$ and $\E\abs{X_i}^3 < \infty, 1 \le i \le n$, then $\forall n \ge 4$,
$$\sup_{x\in\R} \abs{P\left(\frac{S_n}{\sigma_n} \le x \right) - \Phi(x)} \le \frac{2.75}{\sigma^3_n}\sum_{i=1}^n \E\abs{X_i}^3,$$
where $S_n = \sum_{j=1}^n X_j, ~\sigma^2_n = \Var(S_n) = \sum_{i=1}^n \E(X_i^2)$, and $\Phi(\cdot)$ is the cdf of a $\Norm(0,1)$ r.v. \\\bigskip

\textbf{Kolmogorov's 3-Series Theorem}: If $X_1, X_2, \dots$ are ind r.v.\@'s on a psp $(\Omega, \mathcal{F}, P)$, for fixed $c > 0$, define
$$\sum_n P(\abs{X_n} > c), \quad \sum_n \E(X_n^{(c)}), \quad \sum_n \Var(X_n^{(c)}),$$
where $X_n^{(c)} = X_n \indicate{\abs{X_n} \le c}$. Then,
\begin{enumerate}
\item if the 3 series converge for \emph{some} $c > 0$, then $S_n = \sum_{j=1}^n S_j, n\ge1$ converges a.s.$(P)$,
\item if $S_n = \sum_{j=1}^n S_j, n\ge1$ converges a.s.$(P)$, then the 3 series converge for \emph{all} $c > 0$.
\end{enumerate}

\textbf{Corollary}: If $X_1, X_2, \dots$ are ind r.v.\@'s on a psp $(\Omega, \mathcal{F}, P)$ with $\E(X_n) = 0, n \ge 1$, then
\begin{enumerate}
\item if  $\sum_n\big[ \E(X_n^{(c)})^2 + \E\abs{X_n}\indicate{\abs{X_n} > c}\big] < \infty$ for \emph{some} $c > 0$, then $S_n = \sum_{j=1}^n X_j \text{ a.s.}(P)$,
\item if $\sum_n \E\abs{X_n}^{\alpha_n} < \infty$ for \emph{some} $\set{\alpha_n} \subset [1,2]$, then $S_n = \sum_{j=1}^n X_j$ converges a.s.$(P)$.
\end{enumerate}

\subsection*{Useful Inequalities}
For positive $a,b,p$, $(a+b)^p \le 2^p(a^p + b^p)$. \\\medskip

\textbf{Markov's}: if $X$ is a nonnegative r.v.\@ and $a > 0$, then $P(X \ge a) \le \E(X)/a$. \\\medskip

\textbf{Holder's}: If $1/p + 1/q = 1$, then for measurable $f,g$, $\norm{fg}_1 \le \norm{f}_p \norm{g}_q$. \\\medskip

\textbf{Jensen's}: $\forall r \in (0,q)$, $\phi(x) = x^{q/r}$ is convex $\implies \left[\E\abs{X}^q\right]^{1/q} \ge \left[\E\abs{X}^r\right]^{1/r}.$























\end{multicols*}
\end{document}